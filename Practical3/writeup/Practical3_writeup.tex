\documentclass[11pt, oneside]{article}   	% use "amsart" instead of "article" for AMSLaTeX format
\usepackage{geometry}                		% See geometry.pdf to learn the layout options. There are lots.
\usepackage{textcomp}
\usepackage{hyperref}  % TODO: see page 94 of latex book
\geometry{letterpaper}                   		% ... or a4paper or a5paper or ... 
%\usepackage[parfill]{parskip}    		% Activate to begin paragraphs with an empty line rather than an indent
\usepackage{graphicx}				% Use pdf, png, jpg, or eps§ with pdflatex; use eps in DVI mode
								% TeX will automatically convert eps --> pdf in pdflatex		
\usepackage{amssymb}
\usepackage{relsize}

\title{CSCI 181 / E-181 Spring 2014 Practical 3 \\ 
{\large Kaggle Team "Capt. Jinglehiemer"}
}
\author{
  David Wihl\\
  \texttt{davidwihl@gmail.com}
  \and
  Zack Hendlin\\
  \texttt{zgh@mit.edu} 
}
%\date{}							% Activate to display a given date or no date

\begin{document}
\maketitle
\section*{Warm-Up}

\subsection*{Warmup Topic 1}

\subsection*{Warmup Summary}

\section*{Classifying Malicious Software}

\subsection*{Preliminary Data Analysis}

This problem was significantly more complex than the previous practicals due to a number of factors: unbalanced distribution of the targets, sparsity of the data, conversion of hierarchical XML data into rectangular matrices, and unclear value of certain attributes in the data. 

Significant experimentation was required to iterate over feature engineering, classifier choice, permutations of ensembles of classifiers and hyper parameter optimization.

The first step was a significant refactoring of the sample code in order to separate vectorization, classification, validation and submission.

\subsection*{Using Cross-validation}

Very early we created a test harness that enabled testing of $n$ iterations (default 5) over cross-validation sets consisting of 30\%, 20\% and 10\% of the data. This proved very valuable as we experimented. In total, we ran over 200 cross-validation tests each consisting of 15 different passes.

\subsection*{Per File Feature Engineering}

The first vectorizer took simple counts of each different type of system call as well as per-process (e.g. `startreason', `terminationreason', `username', `executionstatus', `applicationtype') and per-file metrics coupled with an unoptimized logistic regression. This enabled us to achieve a 75\% accuracy on the Kaggle submission within just a day or two.

\subsection*{Comparing Classifiers and Tuning Hyperparameters}

We then attempted a number of different classifiers (SVM, kNN, Gradient Boosting). Results are summarized in TODO.

By making multiple cross-validation runs, we tuned each classifiers hyper parameters.

\subsection*{Pruning the Dimensions}

We took two complementary approaches to prune the dimensions.
1. We examined the weights of each of the classifiers to see which dimensions were not producing a signal. Using the mean and standard deviation of the resulting matrix, we noted any features where abs(mean) $<$ 0.001 and std $< $0.01.
2. We exported our Design Matrix to a CSV and then used tools in R to run a linear regression and see the predicted values of the coefficients. 

We then pared back approximately 20 features that provided no value (and ran cross validation again to recheck of course).

\subsection*{Combining Classifiers}

We examined and experimented with several options for combining the classifiers (AdaBoost, Random Forests). We wanted to attempt a novel approach to combining classifiers (using predict\_proba to find what level is $>$98\% is appropriate for a given classifier. Then if the classifier, such kNN, had a high confidence level, it was given the vote. If one classifier had a significantly higher confidence level (determined empirically to be 0.4), it was given the vote. If no classifier had a high confidence, it was given to Logistic Regression as it generally had the lowest error rate.

TODO
Combined LR and kNN. Chose kNN only when it's confidence was 0.4 greater than LR's.


\subsection*{Per System Call Feature Engineering}

We created second, separate model of per system-call metrics, consisting of items such as `successful' attributes. Attributes of type string that had potential value were hashed into a numeric form. This resulted in a design matrix of over 
3000000 rows $\times$ 14 features.

We then applied Logistic Regression to this new design matrix and rechecked, as always with multiple cross-validation passes and again pared back any features that did not have a significant signal.

\subsection*{Quality Check prior to Submission}

After a couple of poor submissions due to bugs in our algorithms, we implemented a sanity checked for submissions. This would useful in real world scenarios to ensure that the predictions are within a reasonable tolerance prior to putting new models into production. The sanity check ensured that all test data had a predictions within the possible values, and then displayed a distribution of distribution to ensure that no unusual skews were occurring.

\subsection*{Exotic Classifiers}

We examined and did preliminary experiments with two different approaches: Exemplar SVM and Theano / Deep Learning. While the reading was interesting, in both cases, the quality of the code and library precluded us from pursuing it further in the time available. We hit a bug in Theano when using the Scipy CSR array. We intend to continue exploring Theano outside the scope of this exercise in the future.


\section*{Conclusion}

This was a valuable exercise in understanding the state of the art in multi-class classification with real-world, sparse hierarchical data. Clearly, \emph{there is no free lunch.}\footnote{Wolpert, Macready, \emph{No Free Lunch Theorems for Optimization} http://ti.arc.nasa.gov/m/profile/dhw/papers/78.pdf}

A more general approach to attacking this type of problem requires more research and investigation. The permutations of features, classifiers and hyper parameters grow geometrically without many significant tools to provide automated means of attacking the problem. This is an area we would be interested in pursuing further research.
\clearpage
We were surprised how little guidance there was in the literature for this type of problem. We scanned through several books, journals, articles and web sites, but found little external guidance that was applicable in this set of circumstances. Clearly many people have attempted to find more general means of automated multi class classification.


\renewcommand{\refname}{Background Reading}

\begin{thebibliography}{1}

 \bibitem{msvm} Multiclass SVM classification http://svmlight.joachims.org/svm\_multiclass.html

 \bibitem{exemplar} Exemplar SVM on github (octave / matlab)
  https://github.com/quantombone/exemplarsvm

 \bibitem{ones} One against all or One against All
  http://hal.archives-ouvertes.fr/docs/00/10/39/55/PDF/cr102875872670.pdf

 \bibitem{sklearn} Classifier comparison:
  http://scikit-learn.org/stable/auto\_examples/plot\_classifier\_comparison.html

  \end{thebibliography}

\end{document}  
